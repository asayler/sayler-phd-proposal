\chapter{Applications}
\label{chap:apps}

As mentioned in previous chapters, the SSaaS model can be utilized
across a range of existing and proposed applications. In each case,
the model allows users to reduce the trust they must place in any
single third party while retaining support for popular use cases
today. I present several potential SSaaS applications in this chapter.

\section{Storage}

One of the primary applications of the SSaaS model is to secure
storage systems, both local and hosted. These applications are all
variations on the previously discussed encrypted data + SSaaS-backed
keys model. In such applications, users handle the encrypting and
decryption of data locally, only sending encrypted data to third
parities or storing it on high risk devices. In order to ensure that
such client-side encryption does not break traditional features
associated with cloud and mobile data storage, clients store the
associated encryption keys with one or more Secret Storage Providers.

\subsection{Cloud File Sync/Storage}

\begin{figure}[t]
  \centering
  \includegraphics[width=150px]{./figs/out/App-FileStore.pdf}
  \caption{SSaaS-Backed Cloud File Storage System}
  \label{fig:apps-filestore}
\end{figure}

Building on the original motivating example in this proposal, using
Dropbox without trusting Dropbox with plain-text user data, we can
construct an SSaaS-backed cloud sync and storage
client. Figure~\ref{fig:apps-filestore} illustrates this
application. As in the general storage case, this applications
involves applying client-side encryption/decryption (e.g. using
AES~\cite{nist2001}) and authentication/verification (e.g. using
CMAC~\cite{dworkin2005}) on every read and write to a third-party
backed data store. The third-party data store holds only encrypted and
authenticated data, ensuring that the third party need not be provided
with the Access (\emph{R}-type) and Manipulation (\emph{W}-type)
capabilities.

In order to ensure that users can still share data with other users
and sync it across devices, all required encryption and authentication
keys are stored with one or more SSPs. When a user wishes to sync data
to a new device, they grant said device access to the necessary keys
via their SSP's management interface. The new device can then download
the encrypted files from the minimally trusted storage provider and
decrypt/verify them using the keys provided by the SSP. Device
authentication can be provided via certificates, shared-secrets
(e.g. passwords), or contextual information as proposed in
Chapter~\ref{chap:custos}. When a user wishes to share data with
another user, they grant the new user access to data via the storage
providers normal sharing mechanisms. They then also grant the new user
access to the necessary keys via the SSPs management mechanisms. The
user can now download the data from the storage provider and decrypt
and verify it with the keys form the SSP. As in the syncing case,
authentication may be performed via a variety of mechanisms, allowing
the data owner to select the authentication primitives best suited for
a given situation.

This type of applications thus overcomes the traditional deficiencies
of secure cloud-based data storage. It minimizes trust in the third
party storage provider by only granting them access to encrypted and
authenticated data. But it also maintains support for the multi-device
and multi-user use cases traditionally associated with cloud-backed
data storage. The SSaaS model allows users to enhance their privacy
and security by reducing exposure to third parties without incurring
undue additional usability burdens or denying access to desirable use
cases.

\subsection{Server Data Encryption}

Beyond consumer-oriented SSaaS-backed encryption systems, there's
strong case for using SSaaS-backed encryption systems for
datacenter-based servers. Leveraging virtual (as well as physical)
servers hosted in cloud data centers is an extremely popular mode of
infrastructure deployment. Unfortunately, as mentioned in
Chapter~\ref{chap:challenges}, the administrator's lack of physical
access to such servers makes it difficult to utilize privacy-enhancing
technologies like Full Disk Encryption (FDE) or file-system level
encryption. In the FDE case, users are generally required to provide
some form of decryption pass-phrase or physical dongle key at boot
time in order to securely bootstrap the system. Similarly, even in the
file-system level encryption case, encryption systems generally
require some form of interactive mechanism to provide the necessary
security pass-phrase bootstrapping the system. Indeed, the lack of
such human-in-the-loop controls hints to serious flaws in the security
of traditional encryption systems. Since administrators generally lack
easy physical access to datacenter-based servers as well as
interactive presence on headless server-oriented machines in general
using traditional encryption systems with must server remains
difficult, if not impossible.

These deficiencies can largely be resolved by relying on SSaaS-backed
encryption systems - either full disk or file-system oriented. Using
SSaaS, a user would configure each server to store their file
encrypting and verification keys with an SSP (or SSPs), configuring
each server to request the keys from the SSP at boot or on access to
an encrypted file. Non-interactive authentication could be provided
using contextual security techniques (e.g. do you expect a server to
be rebooted at a certain time of day each day?) to allow for such
encryption systems to operate largely autonomously. In more sensitive
situations, the SSPs access control system could even keep the user in
the loop as in the traditional encryption use cases by asking them to
provide a pass-phrase or decryption approval via text message or
similar out-of-band real time communication method each time key
access is required.

Such systems would allow users to store sensitive data on servers
while minimizing the degree of trust they must place in third-party
data center hosts. Such servers would store most data in an encrypted
data, ensuring that a physical search of the server or other
interference from the data center host would revel little data. Online
online/in-use data would be decrypt at any given time, and even that
data would be difficult for the data center host to access without
employing high degrees of subterfuge. Even in cases where the data
center host does manage to leverage their control of the underlying
physical systems to trick an SSP into providing decryption keys in
unattended situations, the SSP would still be able to log and audit
the event - making ti extremely difficult for a data center host to
access secure user data in an undetected manner.

Possible implementation of server-based encryption efforts are
discussed further in Chapter~\ref{chap:planned} as part of my
proposed efforts. Such implementation stand to create mechanism to
significantly decrease the amount of trust developers (and by proxy,
their users) must place in the providers of hosted server
infrastructure without significantly raising the cost or overhead of
leveraging such services.

\subsection{Mobile Device Encryption}


\subsection{Personal Data Repository}



\section{Communication}



\section{Authentication}

\subsection{Managed SSH}


\section{Other}

%%  LocalWords:  SSaaS CMAC SSPs SSP's SSP FDE
