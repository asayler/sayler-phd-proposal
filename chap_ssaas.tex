\chapter{Secret Storage as a Service}
\label{chap:ssaas}

As discussed in Chapter~\ref{chap:challenges}, the reliance on third
parties inherent to many popular use cases today poses a number of
privacy and security related challenges. Fortunately, cryptographic
techniques including encryption and authentication provide the
necessary primitives for building a systems that increases the
security and privacy of users by reducing their exposure to third
party-related abuse. Such systems also provide additional security
outside of the traditional third-party risk model by ensure that
systems such as mobile devices that are prone to loss or theft remain
security even when outside the position of their
owners. Unfortunately, cryptography is not merely ``magic fairy dust''
that we can sprinkle on any security or privacy problem to make it
disappear~\cite{smith2003}. Effectively using cryptographic technique
to secure our data involves ensuring that cryptography-employing
security and privacy enhancing solution are designed securely and
usability.

The key to designing secure and usable cryptographic data security
solutions lies in providing secure and flexible secret storage systems
that can be leveraged to manage and access the associated
cryptographic keys protecting any such system. The failure of
traditional cryptographic systems to account for key management has
led many such systems to be unusable, insecure, and/or ill adapted to
modern user cases. I propose the creation of a standardized Secret
Storage as a Service system designed to provide users with the
necessary tools for managing secrets such as cryptographic keys in a
manner that allows for a range of multi-device and multi-user use
cases and that avoids placing high degrees of trust in signaler third
parties. I present the design and justification of such a system in
this chapter.

\section{Architecture}
\label{chap:ssaas:arch}

Secret Storage as a Service (SSaaS) is a cloud architecture where
users utilize dedicated Secret Storage Providers (SSPs) in addition to
the traditional Feature Providers (FPs) like Amazon, Dropbox, Gmail,
or Facebook. An SSP is tasked with the storage of and access control
to a variety of user secrets from cryptographic keys to personal
data. In the normtiize case, users will limit themselves to storing
crtopgphclly protect data on thrid-party FP servers while storign teh
associted crtipgrhic keys proticign such data with a network of
SSPs. This allows SSPs to be selected on the basis of their
trustworthiness while traditional feature providers can be selected on
the basis of their features. The SSaaS model differs from the
traditional cloud model by allowing users to distribute trust across
multiple third parties (or no third parties at all), ensuring that any
single entity need not be fully trusted, while still enabling many
existing cloud use cases.

\subsection{Stored Secrets}

What kind of secrets do we store with an SSP? I believe that users
should really be able to store arbitrary data with any SSP, allowing
open ended secret storage based applications. That said, the SSP model
works best when secrets stored are not inherently sensitive or
revealing when taken alone. This property helps to mitigate the amount
we must trust each SSP. Thus, storing secrets like cryptographic keys
that alone revel no private user data are generally preferable to
storing privacy revealing secrets like plain-text passwords, social
security numbers, etc. I do, however, leave the decision of what to
store with each SSP up to each user and application, and we'll explore
various types of secret storage in Chapter~\ref{chap:apps}.

Another consideration related to what secrets to store with an SSP is
size. I anticipate SSP-based storage selling at a premium price vs
more traditional cloud storage options like S3~\cite{amazon-s3}. This
is due to the difference in priorities between Secret Storage and
generic cloud storage. An SSP is primarily concerned with safeguarding
user secrets and faithfully implementing a user's access control
specifications for each secret. These priorities may very well incur
additional costs not necessary in a more traditional cloud storage
environments: e.g. the need to locate data centers in specific legal
jurisdictions, a greater emphasis of resistant to compelled violations
via legal representation, etc. Thus, it may be desirable for the user
to minimize the amount of data stored with an SSP as a cost
optimization: again making use cases such as storing cryptographic
keys with an SSP while storing the encrypted data with a more
traditional provider desirable.

For these reasons, I feel that storing cryptographic keys with an SSP
is a common enough use case that some SSPs may specifically optimize
for it. Such ``Key Storage as a Service'' (KSaaS) SSPs represent a
subset of the generic SSaaS model.
