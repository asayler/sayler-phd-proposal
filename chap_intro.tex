\chapter{Introduction}
\label{chap:intro}



\section{Overview}

Over the last decade, computing has undergone a monumental shift from
locally stored data on a single personal computer to cloud-based data
storage on a multitude of third party servers. This shift has
generated many benefits: sharing data with other users is trivial,
multi-modal communication between users is easy, and compute devices
are largely ephemeral, easily replaced or transitioned between without
any significant overhead or loss of user data. This transition,
however, has a significant side effect: user data is now largely
stored in manners where it is easily accessible to third parties
beyond the user's immediate control. The shift from locally stored and
controlled user data to third-party user data has a number of
increasingly clear consequences, from increased risk of data
compromise by hackers targeting centralized cloud data stores, to
reduced legal protections from government introspection, to the use of
user statistics in ``big-data'' systems capable of ascertaining more
private information than ever before.

The popularity of the cloud model leads one to believe that most users
are willing to trade the privacy and control afforded by local storage
for the convenience and features cloud-based services provide. Never
the less, a 2014 Pew Research study found that over 90\% of American
adults agree that they have lost control over the data they store in
the cloud, 80\% are concerned about how cloud companies are using
their data, and 70\% are concerned about the manner is which the
government might access their data in the
cloud~\cite{pew-privsec14}. Furthermore, the range of both recently
publicized data leaks at large companies
(e.g.~\cite{apple-icloudleak}) as well as ongoing government
intrusions into cloud-based user data stores
(e.g.~\cite{GreenwaldPrism}) has propelled the debate over user
privacy in the age of the cloud to new levels.

The traditional viewpoint holds that users must choose between either
the conveniences the cloud provides or the privacy and security of
locally stored data. I do not feel that this is true. Instead, I
believe that there are mechanisms that can allow users to retain a
high degree of control over how their data is stored, accessed, and
used while still leveraging a variety of modern third-party
services. The key is disentangling the service we wish to use on the
basis of the features they provide from the third parties we must
trust with control over and access to our data.

To achieve this disentanglement, I propose a new paradigm called
Secret Storage as a Service (SSaaS). In an SSaaS ecosystem, a user
designates one or more trusted SSaaS providers (either self hosted or
third party) with storing and regulating access to their private
secrets (personal information, encryption keys, etc) on their
behalf. Existing technologies and services can than interface with
these SSaaS providers via a standard interface to access user secrets
as allowed by a user-defined set of access control rules. I will
discuss several benefits to this arrangement over the existing
practice of selecting third party services on the basis of their
feature set and implicit providing the same providers with unfettered
access to user data. In particular:

\begin{packed_desc}
\item[No Single Trusted Third Party] \hfill \\ In an SSaaS ecosystem,
  the secret storage provider (SSP) is separate from provider of the
  end-user cloud service (e.g. Dropbox, Gmail, etc) or device
  (e.g. Apple, Lenovo, etc). Furthermore, a user may shard their
  secrets across multiple SSPs, or even host their own SSP. This
  ensures that a user is not giving any single entity control over or
  unfettered access to their data.
\item[Separation of Duties] \hfill \\
  In an SSaaS ecosystem, a user selects a secret storage provider on
  the basis of their trust in that provider while selecting a cloud
  service provider on the basis of the end-user features they
  provide. This allows a user to optimize each selection individually
  instead of having to chose a single provider on the basis of both
  trust and feature set, likely having to sacrifice one in favor of
  the other.
\item[Support for Existing Use Cases] \hfill \\ The SSaaS ecosystem is
  capable of supporting many modern use cases such as sharing data
  with other users or syncing it across a number of personal computing
  devices. Thus, SSaaS allows users to gain privacy and security
  benefits without having to forgo common and popular use cases.
\end{packed_desc}

\section{Motivating Example}
\section{Background}

%%  LocalWords:  SSaaS SSP Lenovo SSPs
