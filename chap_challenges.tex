%\input chap_challenges.tex
\chapter{Challenges to Privacy and Security}
\label{chap:challenges}

\section{Modern Use Cases}

As mentioned in Chapter~\ref{chap:intro}, the last 10+ years have
heralded the rapid expansion of a number of digital data related use
cases. In order to accommodate these use-cases, most modern services
leverage some form of third-party compute or data processing
systems. These third-party systems, however, raise questions about the
privacy and security of user data. In particular, to what degree can
we trust various third-parties with user data and are there mechanisms
that allow us to control or reduce this degree of trust? Before we can
answer these questions, it's important to note that any proposed
solution that fails to support modern desirable use cases is unlikely
to succeed in a market where users are voluntarily turning to
third-party services on the basis of the features they can
provide. Thus, we must start by understanding the predominant modern
use-cases

We can divide these cases into two categories: end-user focused user
cases and developer focused use-cases. End-user use cases are those
that matter to the average consumer. Likewise, developer-focused use
cases are those that matter to back-end developers and service
providers.  Both set of use cases represent requirements that modern
security and privacy enhancing solutions must be able to accommodate
if they are to be widely adopted and used.

\subsection{End-User Use Cases}

Today's end-users expect modern software to support a range of common
behaviors. Chief amongst these are the ability to support the use of
multiple devices per user, the ability to support collaboration and
sharing with other users, and the ability to provide turn-key data
processing capabilities capable of returning useful data to the
end-user.

\subsubsection{Multi-Device}
It's not uncommon for a single user to utilize multiple computing
devices. For example, a user might have a personal laptop, a work
desktop, a smart phone, and a tablet. As such, users expect to be able
access their data from any of their devices. Similarly, many modern
users treat compute devices as essentially disposable. When a user
loses a phone or has a laptop stolen, they still expect to be able to
continue to access their data on any replacement device. The
multi-device and ephemeral-device nature of the average end-user has
lead to the rise of a number of solutions that aim to separate a users
data from any singular device and to ensure that users may access
their data regardless of device. Such solutions can generally be
placed into two groups: services that sync user data between devices
(i.e. sync services) and services that store all user data in a
centralized and globally accessible location and provide a manner for
users to access this from each device (i.e. locker services).

Sync services operate on the premise that user data will be stored
locally on each device, but will also be automatically kept in sync
across multiple devices. They ensure that the users has the same view
of their data regardless of device, even when each device stores
independent, localized copies of this data. Such services generally
accomplish this by providing either a centralized or a decentralized
service that tracks user data on each device and updates data across
all devices when it is add, removed, or modified on any single
device. Sync services are often a desirable solution to the
multi-device data access problem for several reasons:

\begin{packed_desc}
\item[Bandwidth Efficient:] Sync services only require Internet access
  to reflect data changes between devices. The act of reading data can
  be serviced directly from the local copy, avoiding the need to
  consume bandwidth by communicating with an external services or
  another device. This is a desirable quality in situations where
  bandwidth is at a premium and would either be cost prohibitive or
  performance restricting to consume every time data needs to be
  read. Sync services tend to be bandwidth efficient even when
  modifying data since they can cache a series of updates locally,
  only syncing the final state to other devices, as opposed to having
  to translate every individual update across the network.
\item[Offline Support:] As an extension to the previous point, sync
  services are capable of operating in situations where no Internet or
  network connection is available. Since all data access is available
  locally, user may continue to access and modify data even when they
  can not connect to the sync service. The sync service will simply
  wait for the network connection to return and then update any Lola
  changes on other devices. While this can lead to issues when users
  make conflicting modification on multiple devices while offline, in
  general it represents a more graceful failure model then a system
  that requires an Internet connection for any form of data
  access. This also acts as a hedge against the sync service provider
  shutting down, etc; even in such a situation the use will still
  retain local copes of all their data that they could then use to
  bootstrap a new sync service.
\item[No Central Storage:] Since sync services are only concerned with
  syncing changes between devices, it's not inherently necessary for
  such services to store a copy of all user data in a central
  location. This allows such systems to be built using distributed
  device-to-device design when desired. In practice, many sync systems
  do store a copy of all user data in order to facilitate the
  bootstrapping of new devices, but this is not an inherent
  requirement of the sync service architecture.
\end{packed_desc}

One of the main challenges to sync services is their local storage
requirement. While such a requirement allows for some of the benefits
list above, it also limits the total available storage afforded to
each user to the size of their smallest device. In situations where a
user wishes to store more data then can be fit on any single device,
locker services may offer a more desirable solution.

Locker service operate by storing all user data on a central server
and then proving mechanisms for users to access and modify this data
there. Such services are reminiscent of more traditional networked
file systems such as NFS~\cite{Sandberg1985} or
SMB~\cite{microsoft-smb2}. Data lock services store all user data in a
networked location where each individual user device may access it for
the propose of reading, modifying, or deleting data. Since such
systems only have a single logical copy of each file, the user is
presented with a singular view of their data regardless of which
devices they chose to access it from. Such services have several
desirable qualities:

\begin{packed_desc}
\item[No Local Storage:] Locker services store all data in online
  locations, generally atop data-center-based servers. Thus, unlike
  sync services, they require no local storage. This is useful in
  situations where local storage is limited (e.g. on phones) but where
  the user still wishes to have access to a large amount of data
  (e.g. a video collection). Likewise, locker services avoid waiting
  unnecessary local space by creating multiple copies of each file on
  each device. Additionally, the lack of local copies may be desirable
  in situations where local devices are prone to theft or may
  otherwise not pose a reliable and secure platform for the storage of
  all user data.
\item[Single Source of Truth:] Locker services only store a single
  (logical) copy of the data at any time. This ensures that situations
  where the user makes conflicting modifications to a piece of data
  are far less likely then in sync-based solutions. This property can
  also increase the usefulness of such systems in multi-user scenarios
  where more then one person might be accessing and modifying data.
\item[Centralized Control:] Due to the manner in which most locker
  services store their data, such services often provide more
  centralized control point then a potentially detailed sync
  service. Such control may be desirable in corporate environments
  where a single administrator wishes to track user file access,
  sharing, etc.
\end{packed_desc}

One of the main downsides to locker services is their requirement for
always-online access. Thus, in situations where network access is
impossible or where high-bandwidth usage is not practical, locker
services can prevent users from access their data. Thus, locker
services may operate best on local area networks and in other
situations where network bandwidth is reliable and plentifully. in
situations where networks access is not guaranteed, sync services may
offer a more desirable solution.

Today, sync services tend to be the more popular solution for
individual end-users. Such users generally want access to their data
across multiple devices in multiple locations (home, work, in public,
etc) - requiring the data to be available in locations where a
reliable network is not always available. Thus, sync service dominate
multi-device data access solution landscape. Examples of popular
centralized sync services today include Dropbox~\cite{dropbox}
($\approx300$ Million Users~\cite{smith-stats}), Google
Drive~\cite{google-drive} ($\approx240$ Million
Users~\cite{smith-stats}), and Microsoft
OneDrive~\cite{microsoft-onedrive} ($\approx250$ Million
Users~\cite{smith-stats}).  An examples of a decentralized sync
service is BitTorrent Sync~\cite{bittorrent-sync} ($\approx2$ Million
Users~\cite{smith-stats}).

Locker services are also available, especially in situations where
network access is reliable and centralized control desirable:
e.g. business environments. In such situations traditional in-house
networked file system solution may provide the locker
service. Alternatively, there are cloud services that provide users
with online locker-like data access: e.g. system like
OwnCloud~\cite{owncloud} implement the WebDAV
protocol~\cite{Goland1999} for remote file access over the
Internet. Similarly, distributed systems like Least Authority's Simple
Secure Storage Service~\cite{leastauthority-s4} (S4, built atop
Tahoe-LAFS~\cite{Wilcox-O'Hearn2008}) offer Internet-wide multi-device
access to a central data store. Furthermore, some sync services are
also capable of operating more like locker services. Systems like
Microsoft OneDrive allow users to specify which files are copied
locally (and thus available for offline access) and which are stores
only on the server and streamed to the user as
required~\cite{microsoft-onedrive-online}. There have also been a
number of popular special-purpose locker services designed to promote
multi-device access to data, particular in the media space. Systems
like Google Music~\cite{google-music} allow users to upload music
files which they can then access and stream to multiple
devices. Similarly, on the distributed (and generally less-legal)
front, systems like Popcorn Time~\cite{popcorntime} allow users to tap
into each others video libraries to stream content to their own
devices.

Users' ownership of multiple computing devices has lead to a strong
desire for each user to be able to access their data from any
device. In response to this desire, a number of third-party backed
services have sprung up to provide user with either sync or locker
based multi-device file access. Any technology aimed at enhancing the
security or privacy of end user data is going to need to account for
and support the multi-device nature of modern users.

\subsubsection{Multi-User}


\subsubsection{Processing}

IoT

\subsection{Developer Use Cases}

\subsubsection{Third Party Infrastructure}

third party
remote
automation

\section{Failures of Traditional Solutions}

Theft of devices
Provided by third parties

\section{Need for New Solutions}

%%  LocalWords:  OneDrive IoT SMB OwnCloud WebDAV LAFS
