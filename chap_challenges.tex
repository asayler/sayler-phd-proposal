\chapter{Challenges to Privacy and Security}
\label{chap:challenges}

\section{Modern Use Cases}
\label{sec:challenges:usecases}

As mentioned in Chapter~\ref{chap:intro}, the last 10+ years have
heralded the rapid expansion of a number of digital data related use
cases. In order to accommodate these use-cases, most modern services
leverage some form of third-party compute or data processing
systems. These third-party systems, however, raise questions about the
privacy and security of user data. In particular, to what degree can
we trust various third-parties with user data and are there mechanisms
that allow us to control or reduce this degree of trust? Before we can
answer these questions, it's important to note that any proposed
solution that fails to support modern desirable use cases is unlikely
to succeed in a market where users are voluntarily turning to
third-party services on the basis of the features they can
provide. Thus, we must start by understanding the predominant modern
use-cases

We can divide these cases into two categories: end-user focused user
cases and developer focused use-cases. End-user use cases are those
that matter to the average consumer. Likewise, developer-focused use
cases are those that matter to back-end developers and service
providers.  Both set of use cases represent requirements that modern
security and privacy enhancing solutions must be able to accommodate
if they are to be widely adopted and used.

\subsection{End-User Use Cases}

Today's end-users expect modern software to support a range of common
behaviors. Chief amongst these are the ability to support the use of
multiple devices per user, the ability to support collaboration and
sharing with other users, and the ability to provide turn-key data
processing capabilities or other services which act on user data.

\subsubsection{Multi-Device}
It's not uncommon for a single user to utilize multiple computing
devices. For example, a user might have a personal laptop, a work
desktop, a smart phone, and a tablet. As such, users expect to be able
access their data from any of their devices. Similarly, many modern
users treat compute devices as essentially disposable. When a user
loses a phone or has a laptop stolen, they still expect to be able to
continue to access their data on any replacement device. The
multi-device and ephemeral-device nature of the average end-user has
lead to the rise of a number of solutions that aim to separate a users
data from any singular device and to ensure that users may access
their data regardless of device. Such solutions can generally be
placed into two groups: services that sync user data between devices
(i.e. sync services) and services that store all user data in a
centralized and globally accessible location and provide a manner for
users to access this from each device (i.e. locker services).

Sync services operate on the premise that user data will be stored
locally on each device, but will also be automatically kept in sync
across multiple devices. They ensure that the users has the same view
of their data regardless of device, even when each device stores
independent, localized copies of this data. Such services generally
accomplish this by providing either a centralized or a decentralized
service that tracks user data on each device and updates data across
all devices when it is add, removed, or modified on any single
device. Sync services are often a desirable solution to the
multi-device data access problem for several reasons:

\begin{packed_desc}
\item[Bandwidth Efficient:] Sync services only require Internet access
  to reflect data changes between devices. The act of reading data can
  be serviced directly from the local copy, avoiding the need to
  consume bandwidth by communicating with an external services or
  another device. This is a desirable quality in situations where
  bandwidth is at a premium and would either be cost prohibitive or
  performance restricting to consume every time data needs to be
  read. Sync services tend to be bandwidth efficient even when
  modifying data since they can cache a series of updates locally,
  only syncing the final state to other devices, as opposed to having
  to translate every individual update across the network.
\item[Offline Support:] As an extension to the previous point, sync
  services are capable of operating in situations where no Internet or
  network connection is available. Since all data access is available
  locally, user may continue to access and modify data even when they
  can not connect to the sync service. The sync service will simply
  wait for the network connection to return and then update any Lola
  changes on other devices. While this can lead to issues when users
  make conflicting modification on multiple devices while offline, in
  general it represents a more graceful failure model then a system
  that requires an Internet connection for any form of data
  access. This also acts as a hedge against the sync service provider
  shutting down, etc; even in such a situation the use will still
  retain local copes of all their data that they could then use to
  bootstrap a new sync service.
\item[No Central Storage:] Since sync services are only concerned with
  syncing changes between devices, it's not inherently necessary for
  such services to store a copy of all user data in a central
  location. This allows such systems to be built using distributed
  device-to-device design when desired. In practice, many sync systems
  do store a copy of all user data in order to facilitate the
  bootstrapping of new devices, but this is not an inherent
  requirement of the sync service architecture.
\end{packed_desc}

One of the main challenges to sync services is their local storage
requirement. While such a requirement allows for some of the benefits
list above, it also limits the total available storage afforded to
each user to the size of their smallest device. In situations where a
user wishes to store more data then can be fit on any single device,
locker services may offer a more desirable solution.

Locker service operate by storing all user data on a central server
and then proving mechanisms for users to access and modify this data
there. Such services are reminiscent of more traditional networked
file systems such as NFS~\cite{Sandberg1985} or
SMB~\cite{microsoft-smb2}. Data lock services store all user data in a
networked location where each individual user device may access it for
the propose of reading, modifying, or deleting data. Since such
systems only have a single logical copy of each file, the user is
presented with a singular view of their data regardless of which
devices they chose to access it from. Such services have several
desirable qualities:

\begin{packed_desc}
\item[No Local Storage:] Locker services store all data in online
  locations, generally atop data-center-based servers. Thus, unlike
  sync services, they require no local storage. This is useful in
  situations where local storage is limited (e.g. on phones) but where
  the user still wishes to have access to a large amount of data
  (e.g. a video collection). Likewise, locker services avoid waiting
  unnecessary local space by creating multiple copies of each file on
  each device. Additionally, the lack of local copies may be desirable
  in situations where local devices are prone to theft or may
  otherwise not pose a reliable and secure platform for the storage of
  all user data.
\item[Single Source of Truth:] Locker services only store a single
  (logical) copy of the data at any time. This ensures that situations
  where the user makes conflicting modifications to a piece of data
  are far less likely then in sync-based solutions. This property can
  also increase the usefulness of such systems in multi-user scenarios
  where more then one person might be accessing and modifying data.
\item[Centralized Control:] Due to the manner in which most locker
  services store their data, such services often provide more
  centralized control point then a potentially detailed sync
  service. Such control may be desirable in corporate environments
  where a single administrator wishes to track user file access,
  sharing, etc.
\end{packed_desc}

One of the main downsides to locker services is their requirement for
always-online access. Thus, in situations where network access is
impossible or where high-bandwidth usage is not practical, locker
services can prevent users from access their data. Thus, locker
services may operate best on local area networks and in other
situations where network bandwidth is reliable and plentifully. in
situations where networks access is not guaranteed, sync services may
offer a more desirable solution.

Today, sync services tend to be the more popular solution for
individual end-users. Such users generally want access to their data
across multiple devices in multiple locations (home, work, in public,
etc) - requiring the data to be available in locations where a
reliable network is not always available. Thus, sync service dominate
multi-device data access solution landscape. Examples of popular
centralized sync services today include Dropbox~\cite{dropbox}
($\approx300$ Million Users~\cite{smith-stats}), Google
Drive~\cite{google-drive} ($\approx240$ Million
Users~\cite{smith-stats}), and Microsoft
OneDrive~\cite{microsoft-onedrive} ($\approx250$ Million
Users~\cite{smith-stats}).  An examples of a decentralized sync
service is BitTorrent Sync~\cite{bittorrent-sync} ($\approx2$ Million
Users~\cite{smith-stats}).

Locker services are also available, especially in situations where
network access is reliable and centralized control desirable:
e.g. business environments. In such situations traditional in-house
networked file system solution may provide the locker
service. Alternatively, there are cloud services that provide users
with online locker-like data access: e.g. system like
OwnCloud~\cite{owncloud} implement the WebDAV
protocol~\cite{Goland1999} for remote file access over the
Internet. Similarly, distributed systems like Least Authority's Simple
Secure Storage Service~\cite{leastauthority-s4} (S4, built atop
Tahoe-LAFS~\cite{Wilcox-O'Hearn2008}) offer Internet-wide multi-device
access to a central data store. Furthermore, some sync services are
also capable of operating more like locker services. Systems like
Microsoft OneDrive allow users to specify which files are copied
locally (and thus available for offline access) and which are stores
only on the server and streamed to the user as
required~\cite{microsoft-onedrive-online}. There have also been a
number of popular special-purpose locker services designed to promote
multi-device access to data, particular in the media space. Systems
like Google Music~\cite{google-music} allow users to upload music
files which they can then access and stream to multiple
devices. Similarly, on the distributed (and generally less-legal)
front, systems like Popcorn Time~\cite{popcorntime} allow users to tap
into each others video libraries to stream content to their own
devices.

Users' ownership of multiple computing devices has lead to a strong
desire for each user to be able to access their data from any
device. In response to this desire, a number of third-party backed
services have sprung up to provide user with either sync or locker
based multi-device file access. Any technology aimed at enhancing the
security or privacy of end user data is going to need to account for
and support the multi-device nature of modern users.

\subsubsection{Multi-User}

In addition to using multiple compute devices, many users today desire
the ability to share and collaborate with other users. This requires
the ability to share data with other users or to have multiple users
access a central copy of any given piece of data. In response to these
desires, many services today offer various mechanisms for multi-user
sharding and collaboration. Similar to the multi-device use case, the
solution in this space can be roughly grouped into two categories:
distributed services that allows users to share copies of data with
other users and centralized services that allow multiple users access
to a central copy of the data.

In many cases the same solutions discussed previously that enable the
multi-device use case also provide multi-user sharing
capabilities. This is true of sync service like Dropbox or Drive that
not only allow users to sync files amongst their devices, but also
allow them to share files with other users to sync to their
devices. In many ways, this is just an extension of the sync service
model with the extension that now users may also include devices other
then their own ion the sync set. Similarly, locker-style multi-device
solutions often also includes support for multi-user use cases. For
example, traditional networked file systems like NFS provide both
multi-device access and multi-user support. Likewise, systems like
Least Authority's S4 provide primitives for sharing files with
multiple users.

Unlike the sync use case, however, adding support for multi-user
sharing requires providers to provide access control primitives in
addition to basic file syncing primitives. These primitives allow
users to control the manner in which either users may access and use
the shared data, in effect proving users with the necessary tools to
place limes on the degree to which they trust other users. Many
services provide fairly traditional file-like access control schemes
were each user may be granted read and/or write permissions to a
specific piece of data. Data owners can use these permission to craft
access control polices for the group of user with which they wish to
share a given file.

It's also common to see support for various from of multi-user sharing
in a range of hosted services, from social networking apps to
web-based document editors. Social Network systems like
Facebook~\cite{facebook} have extensive support for sharing photos,
videos, status messages, and other user-generated continent. Such
systems also provide the data originator with the ability to place
limits on how data is shared and who it is shared with (although the
effectiveness of such access control settings is often
questionable~\cite{Johnson2012}). Similarly, systems like Google
Docs~\cite{google-docs} or Microsoft Office
Online~\cite{microsoft-officeonline} offer users the ability to
interactively compose and collaborate on document creation and
editing. Such systems are inherently multi-user, generally giving the
user the ability to chose who else can view and edit the document.

Multi-user use cases are a key component of many computing systems
today. As in the multi-device cases, support for multi-user scenarios
is an important component of any privacy and security enhancing
technology. Technologies that lock the user out of such use cases are
unlikely to be popular with most users today.

\subsubsection{Hosted Services and Processing}

In addition to the multi-device and multi-user scenarios users expect
support for today, many user also expect support for various hosted
services and data processing solution. In many ways, this expectation
follows form users' multi-device and multi-user expectations: whereas
the traditional computing model involves users running locally
installed applications for the purpose of processing or interacting
with data, such a model fails to properly account for the multi-device
and multi-user requirements of many modern use cases. Thus, many data
processing services that would have traditionally been installed
locally, are now run as hosted services atop third party
infrastructure. Using such service requires user to be able to share
data with third parties for the purpose of leveraging such
services. Unlike purse multi-device syncing or multi-user sharing
services, data processing service provide some benefit to the user
above and beyond the mere storage, transfer, or sharing of data.

Examples of popular hosted services that interact with user-generated
data include the social networking and document editing solutions
mentioned previously. In both cases, these services take user data and
leverage it to provide an additional benefit to the user: e.g. the
ability to interact and communicate with one's friends or the ability
to create and expand content with a colleague. Other examples of
hosted services include various ``Big Data'' systems that leverage
vast swaths of user data to provide insights into user behavior or
patterns in user actions. For example, ``Internet of Things'' (IoT)
devices are becoming increasingly popular supporting a user's ability
to track things like their day-to-day power consumption~\cite{neurio}
or record their exercise habits~\cite{fitbit}. The data from such
devices in generally passed back to third party processing platforms
where potentially useful insights are drawn form it and returned to
the user. Often such systems leverage their access to wide swaths of
user data to return more useful information then the data form any
single user could provide. It seems like that such services will
continue to grow in popularity as the cost of deploying IoT devices
drops and the collective benefits of access to large data sets
increases.

Modern privacy and security enhancing technologies must account for
the fact that many users may wish to leverage hosted third party
services. Such technologies should ideally provide users with the
ability to share data with such services in a controlled manner and to
transparently audit the manner in which such services are using user
data. Failure to support such scenarios will likely make any given
privacy and security enhancing technology impractical for many current
and future use cases.

\subsection{Developer Use Cases}

Beyond end-user use cases, developers are also heavy users of modern
third party cloud services. As such, there are a number of backend use
cases that would also benefit from privacy and security enhancements
with respect to third party trust. Giving developers the tools to
better protect cloud-baked systems allows them to build more secure
end-user services. Furthermore, developers are often some of the
heaviest users of cloud services, so ensuring that they can adequately
protect their data and services hosted atop third party infrastructure
significantly expands the total number of computing systems
protected. I discuss several common back-end use cases here.

\subsubsection{IaaS and PaaS Infrastructure}

Many production-level systems deployed today run atop third-party
cloud IaaS (Infrastructure as a Service) and PaaS (Platform as a
Service) systems. This fact leads to two main consequences that must
be considered when designing security or privacy enhancing
technologies: lack of full-stack control and the need to scale
dynamically.

In most IaaS and PaaS systems, a developer will not have full control
of the hardware stack on which they are operating from the ground
up. Traditional security and privacy enhancing technologies often rely
on full control of the entire deployment stack, from the raw hardware
all the way through the user-facing software, in order to guarantee
any level of security. In a world where most developer rely on IaaS
and PaaS systems for production deployment, such full stack control is
not possible. The less a trust a developer must place in their IaaS or
PaaS providers, the more direct control they retain over the security
and privacy of their applications. Modern security and privacy
enhancing technology systems should be capable of operating securely
even when the underlying hardware or platform can not be fully vetted
or trusted.

Additionally, cloud-based deployments are often scaled up and down
dynamically as load requires. An application that begins running atop
a single virtual machine may need to scale up to 10s or 100s of
virtual machines as the load increases. Modern cloud platforms are
designed to support such scaling. Thus, any privacy or security
enhancing systems designed to protect such systems must also be cable
of rapid and dynamic scalability. Failure to support such dynamics
will make it difficult for developers to adapt a given security and
privacy enhancing technology in an IaaS/PaaS based world.

\subsubsection{Remote, Headless, and Automated}

It is not uncommon for developers to be working atop remote
infrastructure when utilizing PaaS and IaaS systems. As such, security
and privacy enhancing technologies can should not make assumptions
about a user's ability to physically access a machine. Such physical
access is sometimes required by security enhancing technologies for
the purpose of bootstrapping various encryption systems using
SmartCards, USB drives, etc. Unfortunately the remote, cloud-based
nature of many modern systems do not allow for such physical-access
tied mechanisms.

Similarly, most IaaS-backed servers or PaaS-backed services are
expected to autonomously operate heedlessly (e.g. without a human
operator present) for long stretches of time. Thus, it is not
appropriate to expect developers to be able to provide interactive
keyboard input in support of a security or privacy enhancing
technology. For example, most existing full-disk encryption systems
require a user to enter a pre-boot password each time the system
reboots. Such systems do not operate in a modern cloud-backed
environment and thus are not possible to use in such
scenarios. Furthermore, most modern cloud infrastructure is
bootstrapped automatically by developers using systems like
Puppet~\cite{puppet} or Chef~\cite{chef}. Thus

Assumptions about a developers ability to physically access a machine,
to interactively provide input to a machine, or to manually deploy
each machine no longer hold under the current practice of using
cloud-backed infrastructure. As such. it's important that any
developer-targeted privacy or security enhancing technology avoids
making such assumptions, enduring that it can operate effectively even
atop IaaS and PaaS infrastructure.

\section{Security and Privacy}

The current cloud-computing and computing use case trends raise a
number of security and privacy related questions. To what degree must
we trust each cloud provider to protect our data? How good our cloud
providers at protecting data? What other threats do modern usage
models expose? I explore examples of some of these questions below.

\subsection{Third Party Trust}

Almost all of the use cases discussed in
\S\ref{sec:challenges:usecases} involve ceding some degree of trust to
one or more third parties. This often comes in the form of storing
data or executing computations on their third servers. But to what
degree is this trust well placed? How likely is such trust to lead to
an unintended discourse or manipulation of private data or a related
security failure? Chapter~\ref{chap:trust} discusses these concepts in
more detail. Here I'll present examples of some of the security
failures that can occur related to third party trust.

One of the main forms of trust we place in third parties is not to
intentionally misuse the data we store with them. E.g. Are third
parties suing our data in unexpected and undesirable ways?
Unfortunately, there are a number of examples of such breaches
occurring:

\begin{packed_desc}
\item[Facebook Emotional Contagion Study:] In 2014, it came to light
  that Facebook had engaged in research that involved manipulating
  what users saw in their news feeds in order to study the effects of
  one users emotions on other users~\cite{Goel2014}. The study was
  performed on $\approx700$ users without their explicit knowledge or
  permission. Facebook misused the trust placed in it by it's users by
  leveraging and manipulating their data in unforeseen ways.
\item[Uber User Travel History:] Ride-share app Uber recently made
  headlines when it used the travel-history of a number of it's more
  prompt users to display a live user-location map at a lunch
  party~\cite{Sims2014}. Similarly, the company also used stored user
  travel history to compose a blog post detailing it's ability to
  detect a given user's proclivity for ``one night
  stands''~\cite{Pagliery2014}. In both cases, user leveraged data it
  had about users in manners user did not approve of or intend.
\item[Target Pregnancy Prediction:] In 2012, it became public that
  Target had developed a statistical system for predicting if it's
  shoppers were pregnant based on the kind of items they
  bought. Target would then leverage this data to send customers
  coupons optimized for pregnant individuals. In one case, this
  practice even lead to the outing of a pregnant teenage daughter to
  her previously unaware father~\cite{Hill2012}. Clearly such outcomes
  are not within the realm of what most shopper expect when purchasing
  items at Target.
\end{packed_desc}

While not all of these examples are directly related to a user's use
of cloud services (or fixable through the use of the privacy and
security envying systems proposed in this document), they do show a
range of examples of how third parties can violate the trust placed in
them by their users.

\subsection{Data Breaches}

Beyond direct third party misuse of use data, there is also the risk
on unintentional leaks of data stored with third parties. This may
occur either due to a direct attack on third party or through a
mistake or oversight on the part of the third party. Thus, even if we
trust that a third party won't intentionally misuse our data, we must
still question whether or not they are capable of providing adequate
protection for our data. Today we are generally fully reliant on third
parties to protect the data they store and to faithfully enforce any
data accessing or sharing restrictions we specify.

Unfortunately, there are many examples of third party data breaches
resulting in the unintended release of user data. 2014 alone saw the
release of over 550 million user identities and their associated data
online, representing about 1 in 5 internet users
globally~\cite{SymantecCorporation2014}. While not intentional, such
breaches still call into question the degree to which we should trust
third parties with our data. Examples include:

\begin{packed_desc}
\item[Apple iCloud Celebrity Photo Leak:] In 2014, a number of
  celebrity users of Apple's iCloud data storage
  service~\cite{apple-icloud} were subject to a public release of
  personal photos they had stored with the service, often in various
  stats on undress. This leak was the result of a targeted attacks on
  the corresponding users passwords and iCloud
  accounts~\cite{apple-icloudleak}. These attacks appear to have been
  propagated over several months prior to the public release. While
  this leak was not a result of an overt flaw in Apple's iCloud
  system, the week default security requirements for iCloud accounts
  made it relatively simple for attackers to compromise such accounts
  and steal data.
\item[Anthem and Premera Blue Cross Breaches:] In early 2015 two major
  US health insurance companies were subject to attacks that breached
  their user records, allowing the release of personal, financial, and
  medical information on millions of users~\cite{krebs-anthem,
    krebs-premera}. While the details of the breach are not yet
  public, such attacks demonstrate the risk of trusting a third party
  with the storage of large quantities of sensitive data.
\item[Heartbleed, Shellshock, Etc:] In addition to targeted attacks,
  their parties are also prone to software failures. Prominent
  examples of such failures include Heartbleed~\cite{heartbleed}, a
  flaw in OpenSSL~\cite{openssl} that allows attackers to steal
  private data from many secure servers, and
  Shellshock~\cite{symantec-shellshock}, a GNU Bash~\cite{gnu-bash}
  flaw that allowed user to execute arbitrary code on many web
  servers. Both flaws were widespread and effect large swaths of
  web-connected sites and services, potentially exposing many users to
  attack and data breach's.
\end{packed_desc}

Thus, even if we trust the underling third party provider to properly
store and utilize our data, in many cases the data may still be at
risk for exposure through attack or oversight.

\subsection{Government Intrusion}

Recent events have revealed yet another threat vector that must be
considered when leveraging third party cloud services: the targeting
of such services by various governments for the propose of wide-spread
surveillance. In particular, the NSA leaks revealed by Mr. Edward
Snowden demonstrate the US governments widespread data surveillance
programs targeting popular cloud data
providers~\cite{greenwald-prism}. While such actions, at least in the
US, raise numerous questions of constitutional legality under the 4th
Amendment~\cite{us-constitution-amend4}, it does not at present change
the fact that such searches are known to be occurring. Thus, we are
forced to not only consider our trust of the various third party
providers in the cloud, but also our trust of the governments of the
jurisdictions in which such providers operate.

Numerous examples of privacy-subverting attacks by government actors
have come to light over the previous 5 years. It's worth considering
several of these examples as we evaluate how to increase the security
and privacy guarantees available in the cloud. Notable instances of
government data introspection include:

\begin{packed_desc}
\item[PRISM and MUSCULAR:] The NSA PRISM program was/is a
  FISC-approved~\cite{fisc} system for compelling service providers to
  provide user data to the government~\cite{greenwald-prism}. It is
  believed to be one of the largest mechanisms for the government
  extraction of user data from various could-based services
  (e.g. Google, Yahoo, Microsoft, Etc). Similarly, MUSCULAR was/is a
  joint NSA and British GCHQ effort to intercept and monitor traffic
  within Google's and Yahoo's data center to data center
  networks~\cite{gellman-muscular}. Prior to MUSCULAR's disclosure,
  this intra data center traffic was not generally encrypted, and thus
  was an ideal point for another entity to intercept and monitor user
  data. Both cases demonstrate a concerted government effort to access
  and monitor popular cloud services.
\item[Lavabit:] Lavabit was a private email service with 400,000+
  users premised on the idea that poplar free email service such as
  Gmail lacked adequate security and privacy guarantees. In August,
  2013 Lavabit shuttered in service in response to a US government
  subpoena requiring Lavabit to turn over both all of it's user
  traffic as well as the associated SSL encryption keys necessary to
  decrypt it~\cite{lavabit, levsion-lavabit}. After a legal fight,
  Lavabit founder Ladar Levison was forced to disclose the encryption
  keys for his service. The Lavabit example shows the government's
  willingness to compel operators to aid in the monitoring and
  decryption of user data.
\item[The Great Firewall:] Moving beyond US-based government digital
  surveillance, China has long been known to employ one of the most
  sophisticated web monitoring and content control systems in
  existence~\cite{rsf-china}. The so called ``Great Firewall''
  effectively monitors all Internet traffic traveling in and out of
  China, blocking a range of encrypted services that might be capable
  of subverting such monitoring. Such systems show the willingness of
  some governments to outlaw certain types of technology in the spirit
  of ensure government surveillance efforts are not subverted or
  hindered.
\end{packed_desc}

These examples demonstrate the willingness of many governments to
ensure they can access and monitor user data in the cloud. When
designing security and privacy enhancing systems, we must account for
the fact that services providers may find themselves in potions where
they are compelled to turn over data or where they face large scale
surveillance of all Internet traffic.

\subsection{Physical Security}

The modern usage practices discussed in
\S\ref{sec:challenges:usecases} introduce security issues beyond just
these related to trust and exploit of third party data stores. One of
the key areas where such issues manifest is in the physical security
of modern computing devices. Traditionally the security of a computing
device or any data stored there was rooted on the premise that the
device itself could be kept physically secure: e.g. an attacker would
not posses unrestricted physical access to a device. Modern usage
patterns break this assumption in several ways.

First, the multi-device nature of most users increases the number of
computing devices potentially storing copies of user
data. Furthermore, many modern compute devices are inherently mobile,
easily carried by the user and moved about. The combination of these
facts gratefully reduce the likelihood of a computing device storing
user data being lost or stolen relative to the days when all users
merely owned a single desktop compute located at their place of
residence. As such, it's critical that we design systems that are
resilient to data compromise even when they fall into the hands of
unauthorized actors. The likelihood of user data compromises occurring
due to device loss or theft is far higher today then ever before, and
privacy enhancing solutions must account for this fact.

As mentioned previously, physical control over IaaS-based services is
not generally possible. We are thus forced to run many of our modern
services atop hardware under another parties control. This lack of
physical hardware access has repercussions for modern threat
models. To what extent can the hardware provider interfere or bypass
the security of software that runs on their systems? What abilities do
they have to introspect data stored on their servers? Thus, modern
security and privacy enhancing systems must be designed with the
knowledge the the underlying hardware itself may be
untrustworthy. This is a significant departure form the more
traditional local-hardware threat model.

Both these cases demonstrate that any security and privacy enhancing
systems designed to protect data in either the cloud or atop modern
user devices must content with the fact that the physical security of
the devices on which they operate is not guaranteed. We must design
such systems with the trustworthiness (or lack there of) of such
systems in mind.

\section{Need for New Solutions}

encryption

%%  LocalWords:  OneDrive IoT SMB OwnCloud WebDAV LAFS IaaS PaaS Uber
%%  LocalWords:  iCloud Premera Heartbleed Shellshock OpenSSL NSA
%%  LocalWords:  Snowden th FISC GCHQ MUSCULAR's Lavabit Ladar
%%  LocalWords:  Levison
